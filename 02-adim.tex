\documentclass{beamer}

% This file is a solution template for:

% - Giving a talk on some subject.
% - The talk is between 15min and 45min long.
% - Style is ornate.



% Copyright 2004 by Till Tantau <tantau@users.sourceforge.net>.
%
% In principle, this file can be redistributed and/or modified under
% the terms of the GNU Public License, version 2.
%
% However, this file is supposed to be a template to be modified
% for your own needs. For this reason, if you use this file as a
% template and not specifically distribute it as part of a another
% package/program, I grant the extra permission to freely copy and
% modify this file as you see fit and even to delete this copyright
% notice. 


\mode<presentation>
{
  \usetheme[height=12mm]{Rochester}
  % or ...%

  \setbeamercovered{transparent}
%  % or whatever (possibly just delete it)
}


\usepackage[brazil]{babel}
% or whatever

% or whatever
%\usepackage{graphics}
\usepackage{times}
\usepackage[utf8]{inputenc}
\usepackage{url}

\newcommand{\nektar}{\ensuremath{\mathcal{N}\varepsilon \kappa \tau \alpha r}}
\newcommand{\ordem}[1]{ \ensuremath{\mathcal{O}[#1]}}
\newcommand{\pr}[1]{\ensuremath{ \mathbf{#1}}}    % \pr vem de preto
\newcommand{\etal}{\emph{et al.}}
\newcommand{\jac}[4]{ \ensuremath{ P_{#2}^{#3,#4}(#1) }}
\newcommand{\der}[2]{\ensuremath{ \frac{\partial #1}{\partial #2}}}
\newcommand{\convect}[2]{\ensuremath{ #1 \cdot \nabla #2}}
\newcommand{\R}{\ensuremath{ Re }}
\newcommand{\St}{\ensuremath{ St }}
\newcommand{\cpb}{\ensuremath{ C_{pb}}}
\newcommand{\transf}[3]{\ensuremath{ \int_{-\infty}^\infty #3\: e^{i #2 #1}\: d #2}}
\newcommand\clrms{\ensuremath{\sqrt{\overline{C_L^2}}}}
\newcommand{\epseudo}{\ensuremath{ \epsilon-\text{pseudospectro}}}
\newcommand{\lra}{\ensuremath{\longrightarrow}}

\newcommand{\wt}[1]{\ensuremath{\widetilde{#1}}}
\newcommand{\mcal}[1]{\ensuremath{\mathcal{#1}}}

\newcommand{\ol}[1]{\ensuremath{\overline{#1}}}
\newcommand{\us}{\ensuremath{u_*}}

\newcommand{\p}[1]{\ensuremath{ \mathbf{#1}}}    % \pr vem de preto
\newcommand{\qrq}{\ensuremath{\quad\lra\quad}}
\newcommand{\qqrq}{\ensuremath{\qquad\lra\qquad}}
\newcommand{\pd}{\ensuremath{\partial}}
\newcommand{\bigO}[1]{\ensuremath{\mathcal{O}\left(#1\right)}}


\title{Modelos, Escalas e Semelhança}


\author{Paulo Jabardo}

\titlegraphic{\includegraphics[width=4cm]{figuras/logo-ipt.png}}%}
%   \includegraphics[width=2cm]{fig
%}
\date{24-11-2023}





\begin{document}
\maketitle
\begin{frame}{Análise Dimensional}
  Agora vamos entrar na Análise Dimensional de verdade

  Mas o que fizemos na aula passada?
  \begin{itemize}
  \item Postulamos um modelo matemático - equação diferencial
  \item Identificamos as escalas do problema
  \item Usamos estas escalas para comparar os diferentes termos da equação
  \item De acordo com estimatives, desprezamos alguns termos
  \item Será que precisamos das equações diferenciais?
  \end{itemize}
  NÃO - só comparando estimativas das diferentes contribuições chegamos ao método simplificado.

  Dá para fazer isso de maneira mais abstrata? 
\end{frame}

\begin{frame}{Escoamento ao redor da esfera}
  \[
  F = F(U, D, \rho, \mu)
  \]
\begin{itemize}
\item $F$ - força de arrasto, $N=kg\cdot m/s^2$
\item $U$ - Velocidade da esfera, $m/s$
\item $D$ - diâmetro da esfera, $m$
\item $\rho$ - massa específica do fluido, $kg/m^3$
\item $\mu$ viscosidade do fluido, $Pa\cdot s = kg/(m\cdot s)$
\end{itemize}

Será que podemos usar outras unidades?

\end{frame}

\begin{frame}{Mudando o sistema de unidades}
  Se mudarmos a unidade de comprimento de $m$ para $mm$, multiplicamos o comprimento por um fator $L = 1000$. Os novos valores das grandezas variam de acordo com estas regras:
  
\begin{itemize}
\item $D' \lra L \cdot D$
\item $U' \lra L \cdot U$
\item $\rho' \lra  \rho / L^3$
\item $\mu' \lra \mu / L$
\item $F' \lra L \cdot F$
\end{itemize}
  
\end{frame}

\begin{frame}{Podemos mudar outras unidades}
  \begin{itemize}
  \item Comprimento: L
  \item Tempo: T
  \item Massa: M
  \end{itemize}
Então no novo sistema de unidades:
\begin{itemize}
\item $D' \lra (L) \cdot D$
\item $U' \lra (L/T) \cdot U$
\item $\rho' \lra  (M/L^3) \cdot\rho$
\item $\mu' \lra \{M/(L\cdot T)\}\cdot\mu$
\item $F' \lra (M \cdot L / T^2) \cdot F$
\end{itemize}
  
\end{frame}



\begin{frame}{Sistema de unidades específico para \emph{meu problema}}
  Quero que no novo sistema de unidades, o valor numérico de $D$, $U$, $\rho$ seja $\p{1}$!
  
Para o comprimento,
\[
L \cdot D = 1 \qrq L = \frac{1}{D}
\]
Já para a velocidade,
\[
\frac{L}{T} \cdot U = 1 \qrq T = \frac{U}{D}
\]
para a densidade,
\[
\frac{M}{L^3} \cdot \rho = 1 \qrq M = \frac{1}{\rho D^3}
\]
  
\end{frame}


\begin{frame}{Relação funcional no novo sistema de unidades}
  \[
\frac{M}{L\cdot T} \cdot \mu = \frac{\mu}{\rho U D} = \frac{1}{Re}
\]
\[
\frac{M\cdot L}{T^2} \cdot F = \frac{F}{\rho D^2 U^2} = C_D
\]

$1/Re$ é o valor da viscosidade neste novo sistema de unidades e 
$C_D$ é o valor da força neste novo sistema de unidades onde

$D' = U' = \rho' = 1$

\end{frame}

\begin{frame}{$Re$ e $C_D$ \emph{independem do sistema de unidades}}
\[
\begin{aligned}
Re' = \frac{\rho' U' D'}{\mu'} &=& \frac{\rho (M/L^3) \cdot U (L/T) \cdot D (L) }{\mu M/(L\cdot T)} &=& \frac{\rho U D}{\mu} &=& Re \\
C_D' = \frac{F'}{\rho' D'^2 U'^2} &=& \frac{F (M\cdot L/T^2)}{\rho (M/L^3) \cdot D^2 (L^2) \cdot U^2 (L^2/T^2)} &=& \frac{F}{\rho D^2 U^2} &=& C_D  \\
\end{aligned}
\]
\end{frame}

\begin{frame}{A relação funcional no novo sistema de unidades}
  \[
\frac{F}{\rho D^2 U^2} = F\left(1,1,1, \frac{1}{Re} \right)
\]

ou seja

\[
C_D = C_D(Re)
\]



\end{frame}

\begin{frame}{Parece mágica né?}
  Mas nem tanto...
  \begin{itemize}
  \item Velocidade: \[U = \frac{L}{t}\]
  \item Força: \[F = ma = m\frac{dv}{dt}\]
  \item Densidade \[\rho = \frac{m}{V} = \frac{m}{L^3}\]
  \item Viscosidade \[\tau = \frac{F}{A} = \frac{F}{L^2} = \mu \frac{U}{L}\]
\end{itemize}

\end{frame}

\begin{frame}{Mas e se a física for mais complicada?}
  Escoamento compressível:
  \begin{itemize}
  \item Número de Mach: \[ M = \frac{U}{c} \]
  \item Coeficiente isoentrópico $\gamma$: \[ p \propto \rho^\gamma \]
  \end{itemize}
\end{frame}


\begin{frame}{Unidades}
  Usamos unidades sem pensar muito
  \begin{itemize}
  \item metro
  \item segundo
  \item kilograma
  \item Nós
  \item $km/h$
  \item Newtons
  \item $kg\cdot m /s^2$
\end{itemize}
\end{frame}

\begin{frame}{Unidades simples e derivadas}
\begin{itemize}
\item $m$, $s$, $kg$, $A$, $K$ - Unidades simples
\item $km/h$, $kg/m^3$, $W/mK$, $N$ - Unidades compostas
\end{itemize}

As unidades compostas são formadas a partir de leis físicas combinando
unidades simples:
\begin{itemize}
\item $U = L/t$
\item $F=ma$
\item $\rho = m/L^3$
\end{itemize}

\end{frame}

\begin{frame}{Esta distinção não é tão clara quanto parece: metro}
  Definições do metro
  \begin{itemize}
  \item 1791: $1/10.000.000$ da distância do equador ao polo Norte
  \item 1799: Comprimento de uma barra guardada em algum lugar
  \item 1889: Uma nova barra
  \item 1960: Certo número de comprimentos de onda de uma linha de emissão Kr-86
  \item 1983: Distância que a luz viaja no vácuo durante $1/299.792.458$ segundos
  \item 2019: Mesma coisa mas a definição do segundo mudou
  \end{itemize}
\vspace{0.5cm}
  Não seria o metro uma unidade derivada a partir de 1983???
   
\end{frame}

\begin{frame}{Outro exemplo: Força}
  No SI, o Newton é derivado usando a segunda lei da Newton $F=ma$.

  Mas pega um livro em unidades inglesas:
  \[
  F = \frac{m\cdot a}{g_c}
  \]
  O que é esse $g_c$???
  \vspace{0.5cm}
  
  A força também pode ser definida a partir da gravitação universal:
  \[
  F = G\frac{m_1\cdot m_2}{r^2}
  \]
  Por quê não usar esta relação?
  
\end{frame}

\begin{frame}{Sistema SI}
  \begin{itemize}
  \item Tentativa de facilitar e racionalizar
  \item Hoje não depende de nenhum objeto físico
  \item Usa princípios físicos estabelecidos
  \item A medida que nossa compreensão da física melhora, as definições das unidade podem mudar (e mudaram várias vezes)
  \item Conveniência, confiabilidade e incerteza baixa
  \item Unidades de base e derivadas
  \end{itemize}
\end{frame}

\begin{frame}{Dimensão de uma grandeza}
  \begin{itemize}
    \item Regra de como o valor \emph{numérico} de uma grandeza muda ao se mudar as unidades de base.
    \item Comprimento: unidade reduz por um fator $L$ o valor numérico da grandeza multiplica por $L$
    \item Velocidade: comprimento(L) / tempo(T), valor numérico da grandeza: multiplica por $L/T$
    \item Nossa notação de unidade representa isso de  forma simbólica
\end{itemize}
\end{frame}

\begin{frame}{Dimensão: notação de Maxwell}
    
  Unidades base variam por fatores $L$, $T$, $M$, etc

  As grandezas numéricas variam de acordo com as regras
  \[
  L^a \cdot T^b \cdot M^c\cdot\ldots
  \]

  Força (F): unidade derivada. Em SI a  $N\equiv kg\cdot m / s^2$. Notação de Maxwell:
  \[
  [F] = \frac{M \cdot L}{T^2}
  \]
  
\end{frame}


\begin{frame}{Classes de sistemas de unidades}
  O SI é definido para ser o mais geral possível e conveniente:
  \begin{itemize}
  \item Qualquer problema conhecido pode ser representado (quando isso não acontece: Nobel...)
  \item Existe redundância: temperatura
  \end{itemize}

  Mas e problemas específicos?
  \begin{itemize}
  \item Mecânica: comprimento, massa, tempo $LMT$
  \item Pode ser conveniente trabalha com $LFT$
  \item Estática: $LF$
  \item Calor: $QM\Theta$ (Flogiston...)
  \item Você usa o que for mais conveniente para o teu problema!!!
  \end{itemize}

  \[
    [u] = \phi(L,M,T,\ldots)
    \]
    
  
\end{frame}

\begin{frame}{Porque as dimensões são sempre um monômio?}
  \[
  [u] = \phi(L,M,T,\ldots) = L^\alpha\cdot M^\beta\cdot T^\gamma\ldots
  \]

  Por quê temos unidades da forma
  \[
  \frac{kg\cdot m}{s^2}
  \]
  e não
  \[
  kg + m^2 - s^3
  \]
  ?????

  \emph{TODOS OS SISTEMAS DE UNIDADE SÃO EQUIVALENTES!!!}
\end{frame}

\begin{frame}{Seja uma grandeza $u$}
  \[
[u] = \phi\left( L, M, T, \ldots\right)
\]
Mudando as unidades para sistema 1, o valor de $u$ será
\[
u_1 = u\cdot\phi\left(L_1, M_1, T_1, \ldots\right)
\]
No sistema 2:
\[
u_2 = u\cdot\phi\left(L_2, M_2, T_2, \ldots\right)
\]

Dividindo $u_2/u_1$
\[
\frac{u_2}{u_1} = \frac{ \phi\left(L_2, M_2, T_2, \ldots\right) }{ \phi\left(L_1, M_1, T_1, \ldots\right) }
\]

  
\end{frame}

\begin{frame}{Mas TODOS os sistemas são equivalentes!}
  Considerando o sistema 1 como sistema de unidades original:
  \[
  u_2 = u_1\phi\left(\frac{L_2}{L_1}, \frac{M_2}{M_1}, \frac{T_2}{T_1}, \ldots \right)
\]
ou seja
\[
\frac{ \phi\left(L_2, M_2, T_2, \ldots\right) }{ \phi\left(L_1, M_1, T_1, \ldots\right) } = \phi\left(\frac{L_2}{L_1}, \frac{M_2}{M_1}, \frac{T_2}{T_1}, \ldots \right)
\]
\end{frame}


\begin{frame}{Derivando em relação a $L_2$ e fazendoo $L_2=L_1=L$}
  \[
\frac{\partial_L\phi(L,M,T)}{\phi(L,M,T)} =  \frac{1}{L}\partial_L\phi(1,1,1) = \frac{\alpha}{L} 
\]
ou seja
\[
\phi(L,M,T) = L^\alpha C_1(M,T)
\]
repetindo com $C_1(M,T)$
\[
C_1 = M^\beta C_2(T) \qrq C_2 = C_3T^\gamma\qrq \phi = C_3 L^\alpha M^\beta T^\gamma
\]

\[
C_3 = 1\qrq \phi = L^\alpha M^\beta T^\gamma
\]
\end{frame}

\begin{frame}{Grandezas dependentes e independentes}
  Escoamento incompressível ao redor da esfera:
  \begin{itemize}
  \item Classe $LMT$
  \item Parâmetros: $D$, $U$, $\rho$, $\mu$ e $F$
  \item $[D] = L = L^1 \cdot M^0\cdot T^0$
  \item $[U] = L/T = L^1\cdot M^0 \cdot T^{-1}$
  \item $[\rho] = M/L^3 = L^{-3}\cdot M^1\cdot T^0$
  \item $[\mu] = M/(LT) = L^{-1}\cdot M^1\cdot T^{-1}$
  \item $[F] = ML/T^2$
  \end{itemize}
  \vspace{0.5cm}
  5 parâmetros e 3 dimensões na classe.

  $U$ e $D$ são independentes:
  \[
    [U]^x[D]^y = 1 \lra \frac{L^x}{T^x} L^y = 1\lra y=0, x=0
    \]
    
  
\end{frame}

\begin{frame}{$U$, $D$ e $\rho$ são independentes}
  \[
    [U]^x [D]^y [\rho]^z = 1 \lra \frac{L^x}{T^x} \cdot L^y \cdot \frac{M^z}{L^{3z}}
    \]
    ou seja
\[
\begin{aligned}
  x + y - 3z &= 0\\
  -x &= 0\\
  z &= 0\\
\end{aligned}
\]
Única solução: $x=y=z=0$
   
\end{frame}

\begin{frame}{$\mu$ e $F$ são dependentes}
  \[
    [U]^x [D]^y [\rho]^z [\mu]^w= 1 \lra \frac{L^x}{T^x} \cdot L^y \cdot \frac{M^z}{L^{3z}}\cdot \frac{M^w}{L^w\cdot T^w} = 1
    \]

    \[
\begin{aligned}
  x + y - 3z  &= w\\
  -x &= w\\
  z &= -w\\
\end{aligned}
\]
Admitindo $w=1$,
\[
w=1, x=-1, y=-1, z=-1
\]

\[
  [\mu] = [\rho][U][D] \qrq \frac{[\rho][U][D]}{[\mu]} \equiv 1
  \]
  

\end{frame}

\begin{frame}{Teorema dos $\Pi$s de Buckingham}
  \[
  a = f\left(a_1, a_2, \ldots, a_K, b_1, b_2, \ldots, b_N\right)
  \]
  
\begin{itemize}
\item $a_1$, \ldots, $a_K$ são grandezas com dimensões independentes
\item $b_1, \ldots, b_N$ são grandezas com dimensões dependentes
\item $a$ também tem grandeza dependente
\end{itemize}

\[
\begin{aligned}
  \left[a\right]  &= [a_1]^{\alpha}[a_2]^{\beta} \cdots [a_K]^{\gamma}\\
  \left[b_1\right] &= [a_1]^{\alpha_1}[a_2]^{\beta_1} \cdots [a_K]^{\gamma_1}\\
  \left[b_2\right] &= [a_1]^{\alpha_2}[a_2]^{\beta_2} \cdots [a_K]^{\gamma_2}\\
  \cdots &\\
  \left[b_N\right] &= [a_1]^{\alpha_N}[a_2]^{\beta_N} \cdots [a_K]^{\gamma_N}\\
\end{aligned}
\]


\end{frame}

\begin{frame}{Teorema dos $\Pi$s de Buckingham}
  Calculamos $\alpha$, $\beta$, \ldots, $\gamma$, $\alpha_1$, $\beta_1$, \ldots, $\gamma_1$ , \ldots $\alpha_N$, $\beta_N$, \ldots, $\gamma_N$
  
  As grandezas
  \[
\begin{aligned}
\Pi = \frac{a}{a_1^\alpha a_2^\beta\cdots a_K^\gamma}, \quad
&\Pi_1 = \frac{b_1}{a_1^{\alpha_1} a_2^{\beta_1}\cdots a_K^{\gamma_1}},\\
\Pi_2 = \frac{b_2}{a_1^{\alpha_2} a_2^{\beta_2}\cdots a_K^{\gamma_2}},&\ldots
\Pi_N = \frac{b_N}{a_1^{\alpha_N} a_2^{\beta_N}\cdots a_K^{\gamma_N}}\\
\end{aligned}
\]
\emph{Têm dimensão 1}

\end{frame}


\begin{frame}{Teorema dos $\Pi$s de Buckingham}
  Equação original:



\begin{multline*}
\Pi \cdot \left(a_1^\alpha a_2^\beta\cdots a_K^\gamma\right) =
f\left( a_1, a_2, \ldots, a_K,
\Pi_1 \cdot a_1^{\alpha_1} a_2^{\beta_1}\cdots a_K^{\gamma_1}, \right.\\
\left.\Pi_2 \cdot a_1^{\alpha_2} a_2^{\beta_2}\cdots a_K^{\gamma_2}, 
\ldots, \Pi_N \cdot a_1^{\alpha_N} a_2^{\beta_N}\cdots a_K^{\gamma_N}\right)
\end{multline*}
ou seja
\[
\Pi = \mathcal{F}\left(a_1, a_2, \ldots, a_K, \Pi_1, \Pi_2, \ldots, \Pi_N \right)
\]
mas escolhendo um sistema de unidades onde, numericamente $a_1=a_2=\cdots=a_K=1$:
\[
\Pi = \Pi\left(\Pi_1, \Pi_2, \ldots, \Pi_N \right)
\]


\end{frame}

\begin{frame}{Semelhança}

  No modelo,
\[
\left(\Pi\right)_m = \Pi\left\{\left(\Pi_1\right)_m, \left(\Pi_2\right)_m, \ldots, \left(\Pi_N\right)_m \right\}
\]
No protótipo
\[
\left(\Pi\right)_p = \Pi\left\{\left(\Pi_1\right)_p, \left(\Pi_2\right)_p, \ldots, \left(\Pi_N\right)_p \right\}
\]
Caso,
\[
\left(\Pi_1\right)_p = \left(\Pi_1\right)_m, \: \left(\Pi_2\right)_p = \left(\Pi_2\right)_m, \: \ldots, \: \left(\Pi_N\right)_p = \left(\Pi_N\right)_m
\]
Então
\[
\left(\Pi\right)_p = \left(\Pi\right)_m
\]

\end{frame}

\begin{frame}{Simplificação}
  Relação adimensional:
  \[
\Pi = \Pi\left(\Pi_1, \ldots \Pi_i, \ldots, \Pi_N \right)
\]

Se 

\[
\lim_{\Pi_i\to\infty} \Pi\left(\Pi_1, \ldots \Pi_i, \ldots, \Pi_N \right) = \Pi\left(\Pi_1, \ldots \Pi_{i-1},\Pi_{i+1} \ldots, \Pi_N \right) = const
\]
Podemos simplificar o problema!

\end{frame}


\end{document}

\begin{frame}{}
\end{frame}

\begin{itemize}
\end{itemize}
