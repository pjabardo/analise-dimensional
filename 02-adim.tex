\documentclass{beamer}

% This file is a solution template for:

% - Giving a talk on some subject.
% - The talk is between 15min and 45min long.
% - Style is ornate.



% Copyright 2004 by Till Tantau <tantau@users.sourceforge.net>.
%
% In principle, this file can be redistributed and/or modified under
% the terms of the GNU Public License, version 2.
%
% However, this file is supposed to be a template to be modified
% for your own needs. For this reason, if you use this file as a
% template and not specifically distribute it as part of a another
% package/program, I grant the extra permission to freely copy and
% modify this file as you see fit and even to delete this copyright
% notice. 


\mode<presentation>
{
  \usetheme[height=12mm]{Rochester}
  % or ...%

  \setbeamercovered{transparent}
%  % or whatever (possibly just delete it)
}


\usepackage[brazil]{babel}
% or whatever

% or whatever
%\usepackage{graphics}
\usepackage{times}
\usepackage[utf8]{inputenc}
\usepackage{url}

\newcommand{\nektar}{\ensuremath{\mathcal{N}\varepsilon \kappa \tau \alpha r}}
\newcommand{\ordem}[1]{ \ensuremath{\mathcal{O}[#1]}}
\newcommand{\pr}[1]{\ensuremath{ \mathbf{#1}}}    % \pr vem de preto
\newcommand{\etal}{\emph{et al.}}
\newcommand{\jac}[4]{ \ensuremath{ P_{#2}^{#3,#4}(#1) }}
\newcommand{\der}[2]{\ensuremath{ \frac{\partial #1}{\partial #2}}}
\newcommand{\convect}[2]{\ensuremath{ #1 \cdot \nabla #2}}
\newcommand{\R}{\ensuremath{ Re }}
\newcommand{\St}{\ensuremath{ St }}
\newcommand{\cpb}{\ensuremath{ C_{pb}}}
\newcommand{\transf}[3]{\ensuremath{ \int_{-\infty}^\infty #3\: e^{i #2 #1}\: d #2}}
\newcommand\clrms{\ensuremath{\sqrt{\overline{C_L^2}}}}
\newcommand{\epseudo}{\ensuremath{ \epsilon-\text{pseudospectro}}}
\newcommand{\lra}{\ensuremath{\longrightarrow}}

\newcommand{\wt}[1]{\ensuremath{\widetilde{#1}}}
\newcommand{\mcal}[1]{\ensuremath{\mathcal{#1}}}

\newcommand{\ol}[1]{\ensuremath{\overline{#1}}}
\newcommand{\us}{\ensuremath{u_*}}

\newcommand{\p}[1]{\ensuremath{ \mathbf{#1}}}    % \pr vem de preto
\newcommand{\qrq}{\ensuremath{\quad\lra\quad}}
\newcommand{\qqrq}{\ensuremath{\qquad\lra\qquad}}
\newcommand{\pd}{\ensuremath{\partial}}
\newcommand{\bigO}[1]{\ensuremath{\mathcal{O}\left(#1\right)}}


\title{Modelos, Escalas e Semelhança}


\author{Paulo Jabardo}

\titlegraphic{\includegraphics[width=4cm]{figuras/logo-ipt.png}}%}
%   \includegraphics[width=2cm]{fig
%}
\date{24-11-2023}





\begin{document}
\maketitle
\begin{frame}{Análise Dimensional}
  Agora vamos entrar na Análise Dimensional de verdade

  Mas o que fizemos na aula passada?
  \begin{itemize}
  \item Postulamos um modelo matemático - equação diferencial
  \item Identificamos as escalas do problema
  \item Usamos estas escalas para comparar os diferentes termos da equação
  \item De acordo com estimatives, desprezamos alguns termos
  \item Será que precisamos das equações diferenciais?
  \end{itemize}
  NÃO - só comparando estimativas das diferentes contribuições chegamos ao método simplificado.

  Dá para fazer isso de maneira mais abstrata? 
\end{frame}

\begin{frame}{Escoamento ao redor da esfera}
  \[
  F = F(U, D, \rho, \mu)
  \]
\begin{itemize}
\item $F$ - força de arrasto, $N=kg\cdot m/s^2$
\item $U$ - Velocidade da esfera, $m/s$
\item $D$ - diâmetro da esfera, $m$
\item $\rho$ - massa específica do fluido, $kg/m^3$
\item $\mu$ viscosidade do fluido, $Pa\cdot s = kg/(m\cdot s)$
\end{itemize}

Será que podemos usar outras unidades?

\end{frame}

\begin{frame}{Mudando o sistema de unidades}
  Se mudarmos a unidade de comprimento de $m$ para $mm$, multiplicamos o comprimento por um fator $L = 1000$. Os novos valores das grandezas variam de acordo com estas regras:
  
\begin{itemize}
\item $D' \lra L \cdot D$
\item $U' \lra L \cdot U$
\item $\rho' \lra  \rho / L^3$
\item $\mu' \lra \mu / L$
\item $F' \lra L \cdot F$
\end{itemize}
  
\end{frame}

\begin{frame}{Podemos mudar outras unidades}
  \begin{itemize}
  \item Comprimento: L
  \item Tempo: T
  \item Massa: M
  \end{itemize}
Então no novo sistema de unidades:
\begin{itemize}
\item $D' \lra (L) \cdot D$
\item $U' \lra (L/T) \cdot U$
\item $\rho' \lra  (M/L^3) \cdot\rho$
\item $\mu' \lra \{M/(L\cdot T)\}\cdot\mu$
\item $F' \lra (M \cdot L / T^2) \cdot F$
\end{itemize}
  
\end{frame}



\begin{frame}{Sistema de unidades específico para \emph{meu problema}}
  Quero que no novo sistema de unidades, o valor numérico de $D$, $U$, $\rho$ seja $\p{1}$!
  
Para o comprimento,
\[
L \cdot D = 1 \qrq L = \frac{1}{D}
\]
Já para a velocidade,
\[
\frac{L}{T} \cdot U = 1 \qrq T = \frac{U}{D}
\]
para a densidade,
\[
\frac{M}{L^3} \cdot \rho = 1 \qrq M = \frac{1}{\rho D^3}
\]
  
\end{frame}


\begin{frame}{Relação funcional no novo sistema de unidades}
  \[
\frac{M}{L\cdot T} \cdot \mu = \frac{\mu}{\rho U D} = \frac{1}{Re}
\]
\[
\frac{M\cdot L}{T^2} \cdot F = \frac{F}{\rho D^2 U^2} = C_D
\]

$1/Re$ é o valor da viscosidade neste novo sistema de unidades e 
$C_D$ é o valor da força neste novo sistema de unidades onde

$D' = U' = \rho' = 1$

\end{frame}

\begin{frame}{$Re$ e $C_D$ \emph{independem do sistema de unidades}}
\[
\begin{aligned}
Re' = \frac{\rho' U' D'}{\mu'} &=& \frac{\rho (M/L^3) \cdot U (L/T) \cdot D (L) }{\mu M/(L\cdot T)} &=& \frac{\rho U D}{\mu} &=& Re \\
C_D' = \frac{F'}{\rho' D'^2 U'^2} &=& \frac{F (M\cdot L/T^2)}{\rho (M/L^3) \cdot D^2 (L^2) \cdot U^2 (L^2/T^2)} &=& \frac{F}{\rho D^2 U^2} &=& C_D  \\
\end{aligned}
\]
\end{frame}

\begin{frame}{A relação funcional no novo sistema de unidades}
  \[
\frac{F}{\rho D^2 U^2} = F\left(1,1,1, \frac{1}{Re} \right)
\]

ou seja

\[
C_D = C_D(Re)
\]



\end{frame}

\begin{frame}{Parece mágica né?}
  Mas nem tanto...
  \begin{itemize}
  \item Velocidade: \[U = \frac{L}{t}\]
  \item Força: \[F = ma = m\frac{dv}{dt}\]
  \item Densidade \[\rho = \frac{m}{V} = \frac{m}{L^3}\]
  \item Viscosidade \[\tau = \frac{F}{A} = \frac{F}{L^2} = \mu \frac{U}{L}\]
\end{itemize}

\end{frame}

\begin{frame}{Mas e se a física for mais complicada?}
  Escoamento compressível:
  \begin{itemize}
  \item Número de Mach: \[ M = \frac{U}{c} \]
  \item Coeficiente isoentrópico $\gamma$: \[ p \propto \rho^\gamma \]
  \end{itemize}
\end{frame}

\end{document}

\begin{frame}{}
\end{frame}

\begin{itemize}
\end{itemize}
